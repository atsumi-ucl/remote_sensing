% Options for packages loaded elsewhere
\PassOptionsToPackage{unicode}{hyperref}
\PassOptionsToPackage{hyphens}{url}
\PassOptionsToPackage{dvipsnames,svgnames,x11names}{xcolor}
%
\documentclass[
  letterpaper,
  DIV=11,
  numbers=noendperiod]{scrreprt}

\usepackage{amsmath,amssymb}
\usepackage{lmodern}
\usepackage{iftex}
\ifPDFTeX
  \usepackage[T1]{fontenc}
  \usepackage[utf8]{inputenc}
  \usepackage{textcomp} % provide euro and other symbols
\else % if luatex or xetex
  \usepackage{unicode-math}
  \defaultfontfeatures{Scale=MatchLowercase}
  \defaultfontfeatures[\rmfamily]{Ligatures=TeX,Scale=1}
\fi
% Use upquote if available, for straight quotes in verbatim environments
\IfFileExists{upquote.sty}{\usepackage{upquote}}{}
\IfFileExists{microtype.sty}{% use microtype if available
  \usepackage[]{microtype}
  \UseMicrotypeSet[protrusion]{basicmath} % disable protrusion for tt fonts
}{}
\makeatletter
\@ifundefined{KOMAClassName}{% if non-KOMA class
  \IfFileExists{parskip.sty}{%
    \usepackage{parskip}
  }{% else
    \setlength{\parindent}{0pt}
    \setlength{\parskip}{6pt plus 2pt minus 1pt}}
}{% if KOMA class
  \KOMAoptions{parskip=half}}
\makeatother
\usepackage{xcolor}
\setlength{\emergencystretch}{3em} % prevent overfull lines
\setcounter{secnumdepth}{-\maxdimen} % remove section numbering
% Make \paragraph and \subparagraph free-standing
\ifx\paragraph\undefined\else
  \let\oldparagraph\paragraph
  \renewcommand{\paragraph}[1]{\oldparagraph{#1}\mbox{}}
\fi
\ifx\subparagraph\undefined\else
  \let\oldsubparagraph\subparagraph
  \renewcommand{\subparagraph}[1]{\oldsubparagraph{#1}\mbox{}}
\fi


\providecommand{\tightlist}{%
  \setlength{\itemsep}{0pt}\setlength{\parskip}{0pt}}\usepackage{longtable,booktabs,array}
\usepackage{calc} % for calculating minipage widths
% Correct order of tables after \paragraph or \subparagraph
\usepackage{etoolbox}
\makeatletter
\patchcmd\longtable{\par}{\if@noskipsec\mbox{}\fi\par}{}{}
\makeatother
% Allow footnotes in longtable head/foot
\IfFileExists{footnotehyper.sty}{\usepackage{footnotehyper}}{\usepackage{footnote}}
\makesavenoteenv{longtable}
\usepackage{graphicx}
\makeatletter
\def\maxwidth{\ifdim\Gin@nat@width>\linewidth\linewidth\else\Gin@nat@width\fi}
\def\maxheight{\ifdim\Gin@nat@height>\textheight\textheight\else\Gin@nat@height\fi}
\makeatother
% Scale images if necessary, so that they will not overflow the page
% margins by default, and it is still possible to overwrite the defaults
% using explicit options in \includegraphics[width, height, ...]{}
\setkeys{Gin}{width=\maxwidth,height=\maxheight,keepaspectratio}
% Set default figure placement to htbp
\makeatletter
\def\fps@figure{htbp}
\makeatother
\newlength{\cslhangindent}
\setlength{\cslhangindent}{1.5em}
\newlength{\csllabelwidth}
\setlength{\csllabelwidth}{3em}
\newlength{\cslentryspacingunit} % times entry-spacing
\setlength{\cslentryspacingunit}{\parskip}
\newenvironment{CSLReferences}[2] % #1 hanging-ident, #2 entry spacing
 {% don't indent paragraphs
  \setlength{\parindent}{0pt}
  % turn on hanging indent if param 1 is 1
  \ifodd #1
  \let\oldpar\par
  \def\par{\hangindent=\cslhangindent\oldpar}
  \fi
  % set entry spacing
  \setlength{\parskip}{#2\cslentryspacingunit}
 }%
 {}
\usepackage{calc}
\newcommand{\CSLBlock}[1]{#1\hfill\break}
\newcommand{\CSLLeftMargin}[1]{\parbox[t]{\csllabelwidth}{#1}}
\newcommand{\CSLRightInline}[1]{\parbox[t]{\linewidth - \csllabelwidth}{#1}\break}
\newcommand{\CSLIndent}[1]{\hspace{\cslhangindent}#1}

\KOMAoption{captions}{tableheading}
\makeatletter
\makeatother
\makeatletter
\makeatother
\makeatletter
\@ifpackageloaded{caption}{}{\usepackage{caption}}
\AtBeginDocument{%
\ifdefined\contentsname
  \renewcommand*\contentsname{Table of contents}
\else
  \newcommand\contentsname{Table of contents}
\fi
\ifdefined\listfigurename
  \renewcommand*\listfigurename{List of Figures}
\else
  \newcommand\listfigurename{List of Figures}
\fi
\ifdefined\listtablename
  \renewcommand*\listtablename{List of Tables}
\else
  \newcommand\listtablename{List of Tables}
\fi
\ifdefined\figurename
  \renewcommand*\figurename{Figure}
\else
  \newcommand\figurename{Figure}
\fi
\ifdefined\tablename
  \renewcommand*\tablename{Table}
\else
  \newcommand\tablename{Table}
\fi
}
\@ifpackageloaded{float}{}{\usepackage{float}}
\floatstyle{ruled}
\@ifundefined{c@chapter}{\newfloat{codelisting}{h}{lop}}{\newfloat{codelisting}{h}{lop}[chapter]}
\floatname{codelisting}{Listing}
\newcommand*\listoflistings{\listof{codelisting}{List of Listings}}
\makeatother
\makeatletter
\@ifpackageloaded{caption}{}{\usepackage{caption}}
\@ifpackageloaded{subcaption}{}{\usepackage{subcaption}}
\makeatother
\makeatletter
\@ifpackageloaded{tcolorbox}{}{\usepackage[many]{tcolorbox}}
\makeatother
\makeatletter
\@ifundefined{shadecolor}{\definecolor{shadecolor}{rgb}{.97, .97, .97}}
\makeatother
\makeatletter
\makeatother
\ifLuaTeX
  \usepackage{selnolig}  % disable illegal ligatures
\fi
\IfFileExists{bookmark.sty}{\usepackage{bookmark}}{\usepackage{hyperref}}
\IfFileExists{xurl.sty}{\usepackage{xurl}}{} % add URL line breaks if available
\urlstyle{same} % disable monospaced font for URLs
\hypersetup{
  colorlinks=true,
  linkcolor={blue},
  filecolor={Maroon},
  citecolor={Blue},
  urlcolor={Blue},
  pdfcreator={LaTeX via pandoc}}

\author{}
\date{}

\begin{document}
\ifdefined\Shaded\renewenvironment{Shaded}{\begin{tcolorbox}[borderline west={3pt}{0pt}{shadecolor}, frame hidden, enhanced, breakable, boxrule=0pt, sharp corners, interior hidden]}{\end{tcolorbox}}\fi

\hypertarget{week-1}{%
\chapter{Week 1}\label{week-1}}

\hypertarget{summary}{%
\section{Summary}\label{summary}}

Amongst the many points addressed in the lectures were: 1. Earth
Observation data uses electromagnetic radiation reflected or emitted
from the object or area of interest that travels at the speed of light
(because electromagnetic radiation is light itself!?!?). and 2. Some
objects /electromagnetic radiation in a certain spectrum which cannot be
viewed by human eyes can be detected by satellite sensors to allow human
eyes to see them. Because colours are nothing but light, and light is
nothing but electromagnetic radiation and electromagnetic radiation
consists of different waves which is part of the electromagnetic
spectrum ?!?! (this blows my mind as i write. Am I supposed to have
known this as someone who finished primary education a number of years
ago?)

eg. leaves of the trees (or vegetation) can be viewed in the
near-infrared light.

Why leaves green? Wavelengths in the green region of the electromagnetic
spectrum are reflected by pigments in the leaf' (NASA Science). That's
why we see green leaves. But also `a plant with more chrlorophyll
(pigment) will reflect more near-infrared energy than an unhealthy
plant' (NASA Science). So this is why with satellite imagery, we can use
both the visible and invisible near-infrared spectrum to study
vegetation.

SCATTERING: Why is the sky black on the moon? (no atmosphere) Why is the
sky blue? (short waves scattered) Why is the sky red when the sun goes
down?

Why is the ocean blue? - water absorbs the red yellow and green waves
but reflects /scatters blue.

Finally, there are two types of sensors to record electromagnetic
radiation.

1. Passive sensors record EMR reflected or emitted from the terrain.

2. Active sensors such as LiDAR (on aircrafts), SAR or RADAR emit
electromagnetic waves and record the amount of radiant flux (energy per
unit of time) that travelled back to the sensors. These are used to
develop the DEM, digital elevation model, using the speed of travel to
measure the distance between the point on the terrain and the location
of the sensor. SAR can see through clouds.

\hypertarget{application}{%
\section{Application}\label{application}}

In the field of global public health, malaria or infectious disease
epidemiologists have been the forerunner in using geospatial data
including Earth Observation data. For example, this paper by Brousse et
al. (2020) uses the urban classification based on the Local Climate Zone
map which is developed based on satellite imagery.

So far,

\hypertarget{reflection}{%
\section{Reflection}\label{reflection}}

\hypertarget{week-2}{%
\chapter{Week 2}\label{week-2}}

\hypertarget{summary-1}{%
\section{Summary}\label{summary-1}}

\hypertarget{application-1}{%
\section{Application}\label{application-1}}

\hypertarget{reflection-1}{%
\section{Reflection}\label{reflection-1}}

\hypertarget{week-3}{%
\chapter{Week 3}\label{week-3}}

\hypertarget{summary-2}{%
\section{Summary}\label{summary-2}}

\hypertarget{application-2}{%
\section{Application}\label{application-2}}

\hypertarget{reflection-2}{%
\section{Reflection}\label{reflection-2}}

\begin{enumerate}
\def\labelenumi{\arabic{enumi}.}
\item
  Week 1

  Summary

  Application

  Reflection
\item
  Week 2
\item
  Week 3
\item
  Week 4
\item
\end{enumerate}

\hypertarget{refs}{}
\begin{CSLReferences}{1}{0}
\leavevmode\vadjust pre{\hypertarget{ref-Brousse2020}{}}%
Brousse, O, S Georganos, M Demuzere, S Dujardin, M Lennert, C Linard, R
W Snow, W Thiery, and N P M van Lipzig. 2020. {``Can We Use Local
Climate Zones for Predicting Malaria Prevalence Across Sub-Saharan
African Cities?''} \emph{Environmental Research Letters} 15 (12):
124051. \url{https://doi.org/10.1088/1748-9326/abc996}.

\end{CSLReferences}



\end{document}
